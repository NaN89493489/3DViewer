\documentclass[12pt, a4paper]{report}
\usepackage{setspace}
\onehalfspacing
\title{3DViewer-v1.0}
\author{Rooneele Ayrmidok}
\begin{document}
\section{Intro}
3DViewer-v1.0 is a simple viewer for 3D models. 

\section{The example of supported .obj file format}
This version of the viewer partially suports .obj file format.
In this project we were to implement only vertices and surfaces list support.


  \# List of geometric vertices, with (x, y, z [,w]) coordinates, w is optional and defaults to 1.0.
  
  v 0.123 0.234 0.345 1.0
  
  v ...
  
  ...
  
  \# Polygonal face elements
  
  f v1 v2
  f v1 v2 v3 v4

  ...
  
Different number of verteces in polygonal face are supported.

\section{Basic functions}
The program provides the ability to:
- Load a wireframe model from an obj file (vertices and surfaces list support only).
- Translate the model by a given distance in relation to the X, Y, Z axes.
- Rotate the model by a given angle relative to its X, Y, Z axes.
- Scale the model by a given value.

To perform these operations user is to enter parameters of any
transformation in the panel on the left.

\section{Bonus functions}
- The program allows customizing the type of projection (parallel and central)
- The program allows setting up the type (solid, dashed), color and thickness of the edges, display method (none, circle, square), color and size of the vertices
- The program allows choosing the background color
- All settings are saved by clicking on the "Save configuration" button, and then you can load the custom settings by clicking on the "Load configuration" button.
- To set the default values, click the "Set the configuration to the default value" button.
- The program allows saving the captured (rendered) images as bmp and jpeg files.
- The program allows recording small screencasts by a special button - the current custom affine transformation of the loaded object into gif-animation (10fps, 5s)

\end{document}
